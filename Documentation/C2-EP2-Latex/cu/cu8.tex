% \IUref{IUAdmPS}{Administrar Planta de Selección}
% \IUref{IUModPS}{Modificar Planta de Selección}
% \IUref{IUEliPS}{Eliminar Planta de Selección}

% 


% Copie este bloque por cada caso de uso:
%-------------------------------------- COMIENZA descripción del caso de uso.

%\begin{UseCase}[archivo de imágen]{UCX}{Nombre del Caso de uso}{
%--------------------------------------
	\begin{UseCase}{CU8}{Generar cita}{
		El director, un nutriólogo, profesor o médico pueden generar una cita con los tutores.
	}
		\UCitem{Versión}{\color{Gray}0.1}
		\UCitem{Autor}{\color{Gray}Diego Rosas Cruz}
		\UCitem{Supervisa}{\color{Gray}Ulises Vélez Saldaña.}
		\UCitem{Actor}{\hyperlink{director}{director}, \hyperlink{medico}{Medico}, \hyperlink{nutriologo}{Nutriologo}, \hyperlink{docente}{profesor}}
		\UCitem{Propósito}{Se necesita generar una cita cuando ocurra alguna incidencia o cuando se necesite informar a los tutores sobre algo relacionado a la guardería y sus hijos.}
		\UCitem{Entradas}{nombre del niño, nombre del tutor, fecha y hora de la cita}
		\UCitem{Origen}{Pantalla}
		\UCitem{Salidas}{MSGX.}
		\UCitem{Destino}{IUX}
		\UCitem{Precondiciones}{El niño y su tutor deben estar registrados.
  El actor que genere la cita debe haber iniciado sesión en el sistema.
  }
		\UCitem{Postcondiciones}{Se generará una nueva cita.}
	   \UCitem{Errores}{
            \begin{itemize}
                \item Si el actor que intenta generar la cita ya tiene otro cita en esa fecha y hora el sistema pedirá que ingrese otra fecha y hora.
                \item No se llenaron los datos correctamente; el sistema mostrara en que campos se tiene error y solicitara su corrección.
            \end{itemize}
            }
		\UCitem{Tipo}{Caso de uso primario}
		\UCitem{Observaciones}{ninguna}
	\end{UseCase}
%--------------------------------------
%--------------------------------------
	\begin{UCtrayectoria}
		\UCpaso[\UCactor] Accede al sistema.
		\UCpaso muestra la \IUref{IUX}{Pantalla principal}.
		\UCpaso[\UCactor] solicita generar cita
		\UCpaso muestra la \IUref{IUX}{Generar cita}.
		\UCpaso[\UCactor] Ingresa nombre del niño, del tutor, fecha y hora para la cita.
		\UCpaso verifica  que en la fecha y hora indicadas el actor no tenga otra cita.\Trayref{A}.
        \UCpaso Verifica que todos los datos estén llenados correctamente \Trayref{B}.
		\UCpaso registra una nueva cita y muestra al actor el mensaje de éxito {\bf MSGX}:.
	\end{UCtrayectoria}
 %--------------------------------------		
		\begin{UCtrayectoriaA}{A}{El actor ya tiene una cita en esa fecha y hora.}
			\UCpaso Muestra el Mensaje {\bf MSGX}``Esa fecha y hora ya están ocupadas, por favor ingrese una fecha y hora distintas.''.
			\UCpaso Continua en el paso 7 del caso de uso.
		\end{UCtrayectoriaA}
  %--------------------------------------
		\begin{UCtrayectoriaA}{B}{Algun dato del formulario incorrecto}
			\UCpaso Muestra el Mensaje {\bf MSGX}``Revise los datos ingresados.''.
                \UCpaso Subraya de rojo los campos que tienen problemas con los datos ingresados.
			\UCpaso Continua en el paso 5 del caso de uso.
		\end{UCtrayectoriaA}

%-------------------------------------- TERMINA descripción del caso de uso.