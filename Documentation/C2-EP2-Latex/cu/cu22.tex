% \IUref{IUAdmPS}{Administrar Planta de Selección}
% \IUref{IUModPS}{Modificar Planta de Selección}
% \IUref{IUEliPS}{Eliminar Planta de Selección}

% 


% Copie este bloque por cada caso de uso:
%-------------------------------------- COMIENZA descripción del caso de uso.

%\begin{UseCase}[archivo de imágen]{UCX}{Nombre del Caso de uso}{
%--------------------------------------
	\begin{UseCase}{CU22}{Generar reporte}{
		El profesor genera un reporte de un niño.
	}
		\UCitem{Versión}{\color{Gray}0.1}
		\UCitem{Autor}{\color{Gray}Diego Rosas Cruz}
		\UCitem{Supervisa}{\color{Gray}Ulises Vélez Saldaña.}
		\UCitem{Actor}{\hyperlink{Profesor}{Capital humano}}
		\UCitem{Propósito}{Se necesita llevar registro de las actividades de los niños, sus cambios de ropa, ingestas y evacuaciones, así como las incidicencias que ocurrieron durante el día.}
		\UCitem{Entradas}{Asistencia, evacuaciones, ingesta, cambios de ropa, información adicional, actividades, tareas.}
		\UCitem{Origen}{Pantalla}
		\UCitem{Salidas}{MSGX.}
		\UCitem{Destino}{IUX}
		\UCitem{Precondiciones}{El niño debe estar registrado.
  El profesor que genere el reporte debe haber iniciado sesión.
  }
		\UCitem{Postcondiciones}{Se generará un nuevo reporte en el sistema.}
	   \UCitem{Errores}{
            \begin{itemize}
                \item No se llenaron los datos correctamente; el sistema mostrara en que campos se tiene error y solicitara su corrección.
            \end{itemize}
            }
		\UCitem{Tipo}{Caso de uso primario}
		\UCitem{Observaciones}{ninguna}
	\end{UseCase}
%--------------------------------------
%--------------------------------------
	\begin{UCtrayectoria}
		\UCpaso[\UCactor] Accede al sistema.
		\UCpaso muestra la \IUref{IUX}{Pantalla de profesores}.
		\UCpaso[\UCactor] solicita generar reporte.
		\UCpaso muestra la \IUref{IUX}{Generar Reporte}.
		\UCpaso[\UCactor] Ingresa los datos del niño.
        \UCpaso Verifica que todos los datos estén llenados correctamente \Trayref{A}.
		\UCpaso registra un reporte nuevo y muestra al actor el mensaje de éxito {\bf MSGX}:.
	\end{UCtrayectoria}
		\begin{UCtrayectoriaA}{A}{Algun dato del formulario incorrecto}
			\UCpaso Muestra el Mensaje {\bf MSGX}``Revise los datos ingresados.''.
                \UCpaso Subraya de rojo los campos que tienen problemas con los datos ingresados.
			\UCpaso Continua en el paso 5 del caso de uso.
		\end{UCtrayectoriaA}

%-------------------------------------- TERMINA descripción del caso de uso.