% \IUref{IUAdmPS}{Administrar Planta de Selección}
% \IUref{IUModPS}{Modificar Planta de Selección}
% \IUref{IUEliPS}{Eliminar Planta de Selección}

% 


% Copie este bloque por cada caso de uso:
%-------------------------------------- COMIENZA descripción del caso de uso.

%\begin{UseCase}[archivo de imágen]{UCX}{Nombre del Caso de uso}{
%--------------------------------------
	\begin{UseCase}{CU3}{Registrar niño}{
		Capital humano llenará los datos de un niño para registrarlo en el sistema.
	}
		\UCitem{Versión}{\color{Gray}0.1}
		\UCitem{Autor}{\color{Gray}Diego Rosas Cruz}
		\UCitem{Supervisa}{\color{Gray}Ulises Vélez Saldaña.}
		\UCitem{Actor}{\hyperlink{capitalHumano}{Capital humano}}
		\UCitem{Propósito}{Es necesario tener los datos del niño en caso de que ocurra una incidencia, además los dos se requerien para tener al niño registrado en una sala, así como llevar un registro de todo lo relacionado con el niño.}
		\UCitem{Entradas}{Nombre, dirección, fecha de nacimiento.}
		\UCitem{Origen}{Pantalla}
		\UCitem{Salidas}{MSGX.}
		\UCitem{Destino}{IUX}
		\UCitem{Precondiciones}{El niño no debe estar registrado}
		\UCitem{Postcondiciones}{Habrá un niño nuevo registrado en el sistema.}
		\UCitem{Errores}{
            \begin{itemize}
                \item Si el niño que intenta registar ya existe el sistema mostrará el mensaje de error y solicitará otra vez los datos.
                \item No se llenaron los datos correctamente; el sistema mostrara en que campos se tiene error y solicitara su corrección.
            \end{itemize}
            }
		\UCitem{Tipo}{Caso de uso primario}
		\UCitem{Observaciones}{ninguna}
	\end{UseCase}
%--------------------------------------
%--------------------------------------
	\begin{UCtrayectoria}
		\UCpaso[\UCactor] Accede al sistema.
		\UCpaso muestra la \IUref{IU5}{Pantalla principal Capital humano}.
		\UCpaso[\UCactor] solicita registrar un niño.
		\UCpaso muestra la \IUref{IUX}{Registrar niño}.
		\UCpaso[\UCactor] llena todos los datos solicitados para registrar un niño.
		\UCpaso verifica que ese niño no esté registrado\Trayref{A}.
		\UCpaso Verifica que todos los datos esten llenados correctamente \Trayref{B}.
		\UCpaso registra el nuevo niño y muestra al actor el mensaje de éxito {\bf MSGX}:.
	\end{UCtrayectoria}
 %--------------------------------------		
		\begin{UCtrayectoriaA}{A}{El niño ya existe}
			\UCpaso Muestra el Mensaje {\bf MSGX}``Ese niño ya está registrado, por favor revise los datos.''.
			\UCpaso Continua en el paso 7 del caso de uso.
		\end{UCtrayectoriaA}
		
%--------------------------------------
		\begin{UCtrayectoriaA}{B}{Algun dato del formulario incorrecto}
			\UCpaso Muestra el Mensaje {\bf MSGX}``Revise los datos ingresados.''.
                \UCpaso Subraya de rojo los campos que tienen problemas con los datos ingresados.
			\UCpaso Continua en el paso 5 del caso de uso.
		\end{UCtrayectoriaA}

%-------------------------------------- TERMINA descripción del caso de uso.