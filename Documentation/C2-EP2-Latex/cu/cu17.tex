% \IUref{IUAdmPS}{Administrar Planta de Selección}
% \IUref{IUModPS}{Modificar Planta de Selección}
% \IUref{IUEliPS}{Eliminar Planta de Selección}

% 


% Copie este bloque por cada caso de uso:
%-------------------------------------- COMIENZA descripción del caso de uso.

%\begin{UseCase}[archivo de imágen]{UCX}{Nombre del Caso de uso}{
%--------------------------------------
	\begin{UseCase}{CU17}{Registrar evento escolar}{
		El director registrará un evento escolar.
	}
		\UCitem{Versión}{\color{Gray}0.1}
		\UCitem{Autor}{\color{Gray}Diego Rosas Cruz}
		\UCitem{Supervisa}{\color{Gray}Ulises Vélez Saldaña.}
		\UCitem{Actor}{\hyperlink{capitalHumano}{Capital humano}}
		\UCitem{Propósito}{Es necesario registrar el evento escolar para informar a los padres.}
		\UCitem{Entradas}{fecha del evento, nombre del evento, hora de inicio y fin del evento.}
		\UCitem{Origen}{Pantalla}
		\UCitem{Salidas}{MSGX.}
		\UCitem{Destino}{IUX}
		\UCitem{Precondiciones}{El director debe haber iniciado sesión.
  }
		\UCitem{Postcondiciones}{Se registrará un evento.}
	   \UCitem{Errores}{
            \begin{itemize}
                \item Si el director elige una cita que ya está ocupada para otro evento el sistema le pedirá que escoja otra cita.
                \item No se llenaron los datos correctamente; el sistema mostrara en que campos se tiene error y solicitara su corrección.
            \end{itemize}
            }
		\UCitem{Tipo}{Caso de uso primario}
		\UCitem{Observaciones}{ninguna}
	\end{UseCase}
%--------------------------------------
%--------------------------------------
	\begin{UCtrayectoria}
		\UCpaso[\UCactor] Accede al sistema.
		\UCpaso muestra la \IUref{IUX}{Pantalla principal}.
		\UCpaso[\UCactor] solicita registrar evento
		\UCpaso muestra la \IUref{IUX}{Registrar evento}.
		\UCpaso[\UCactor] Ingresa el nombre, fecha, hora de inicio y fin del evento. 
		\UCpaso verifica  que en la fecha y hora indicadas no haya otro evento.\Trayref{A}.
        \UCpaso Verifica que todos los datos estén llenados correctamente \Trayref{B}.
		\UCpaso registra un nuevo evento y muestra al director el mensaje de éxito {\bf MSGX}:.
	\end{UCtrayectoria}
 %--------------------------------------		
		\begin{UCtrayectoriaA}{A}{Ya hay un evento registrado en esa fecha.}
			\UCpaso Muestra el Mensaje {\bf MSGX}``Esa fecha y horas ya están ocupadas, por favor ingrese otras.'.
			\UCpaso Continua en el paso 7 del caso de uso.
		\end{UCtrayectoriaA}
  %--------------------------------------
		\begin{UCtrayectoriaA}{B}{Algún dato del formulario incorrecto}
			\UCpaso Muestra el Mensaje {\bf MSGX}``Revise los datos ingresados.''.
                \UCpaso Subraya de rojo los campos que tienen problemas con los datos ingresados.
			\UCpaso Continua en el paso 8 del caso de uso.
		\end{UCtrayectoriaA}

%-------------------------------------- TERMINA descripción del caso de uso.