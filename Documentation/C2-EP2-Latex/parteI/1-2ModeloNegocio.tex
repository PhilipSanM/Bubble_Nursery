
% ----------------------------------------------------
\chapter{Modelado del negocio.}

El modelado del negocio es una etapa fundamental en el análisis y diseño de sistemas, ya que permite comprender a fondo el funcionamiento y las necesidades del negocio en el que se enmarca el proyecto. A través del modelado del negocio, se logra una representación clara y estructurada de los procesos, reglas y conceptos fundamentales que rigen las operaciones y objetivos del negocio.//

En esta sección, se analizarán las reglas del negocio, que son las directrices y restricciones establecidas para guiar las actividades y decisiones dentro del contexto de la guardería. Estas reglas pueden incluir políticas, procedimientos, normativas legales o cualquier otro tipo de regulación que influya en la forma en que se llevan a cabo las operaciones y se gestionan los recursos en la guardería.

Asimismo, se abordarán los términos del negocio, que son los conceptos y vocabulario específico utilizados en el ámbito de la guardería. Estos términos representan elementos clave de la guardería y su correcta definición y comprensión son esenciales para establecer una comunicación efectiva entre los diferentes actores y equipos involucrados en el proyecto de la guardería.//

La guardería “Burbujas” es una institución educativa de mediana escala ubicada en una zona suburbana. La guardería ofrece servicios de cuidado y formación temprana para niños y niñas de entre 3 meses y 5 años de edad, con un enfoque en la seguridad y el desarrollo saludable de los niños. La guardería cuenta con un equipo de profesionales altamente capacitados y comprometidos con el bienestar de los niños y sus familias, y busca ofrecer un servicio de alta calidad. La guardería busca enfocarse en establecer una comunicación efectiva con los padres, para mantenerlos informados sobre las actividades y el desarrollo de sus hijos. También se cuenta con servicios de nutriólogo, que se encarga de elegir las comidas de los niños, y médico general.

% ----------------------------------------------------
\section{Terminos del negocio.}
\begin{itemize}
\item \textbf{Alumno}: Un infante que queda al cuidado del personal en la guardería.
\item \textbf{Empleado}: Cualquier persona que labore en la guardería.
\item \textbf{Maestro}: (es tipo de empleado) se encarga de cuidar y enseñar a los alumnos que tenga asignados.
\item \textbf{Director}: Se encarga de la administración de la guardería, elige los días inhábiles o aquellos en los que se llevarán a cabo actividades para que los padres sean avisados.
\item \textbf{Médico}: (es un tipo de empleado) se encarga de atender a los niños y personal de la guardería que sufran algún imprevisto.
\item \textbf{Nutriólogo}: (es un tipo de empleado) registra mensualmente el menú de las comidas diarias de los niños.
\item \textbf{Capital Humano}: (es un tipo de empleado) se encargan de registrar y modificar la información de los niños.
\item \textbf{Padre}: Se refiere a la persona que va a encargar a su hijo en la guardería.
\item \textbf{Checador}: Reloj asociado al atributo: Hora de entrada y salida del empleado. Frecuencia de lectura: una vez al día para la entrada y una vez al día para la salida durante los días laborales.
\end{itemize}